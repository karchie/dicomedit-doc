\documentclass[handout]{beamer}
% \documentclass{beamer}  % include pauses for talk slides
\usetheme{Singapore}
\usepackage{color,enumerate,fancyvrb,hyperref,verbatim}
\title[DicomEdit]{DicomEdit\\A language for DICOM metadata modification}
\author{Kevin A. Archie}
\institute{Washington University Neuroinformatics Research Group}
\date{12 September 2011}

\mathchardef\mhyphen="2D

\begin{document}

\AtBeginSubsection[]
{
  \begin{frame}<beamer>
    \frametitle{Overview}
    \tableofcontents[currentsection,currentsubsection]
  \end{frame}
}

\begin{frame}
\titlepage
\end{frame}

\section{Introduction}

\begin{frame}{What is DicomEdit?}
\begin{itemize}
\pause \item Small language for specifying changes to DICOM metadata
\pause \item a.k.a. ``anon script'' language
\pause \item Interpreter is a modular component:
\url{http://hg.xnat.org/dicomedit}
\pause \item Similar in spirit (but not syntax) to MIRC Anonymizer language
\end{itemize}
\end{frame}

\subsection{History}
\begin{frame}{History}
\begin{itemize}
\pause \item DicomBrowser (late 2006; v1.0 released 1 March 2007)
\pause \item Scriptable modifications (in DicomBrowser v1.0)
\pause \item DicomSummarize/DicomRemap (first command-line DicomEdit, early 2008)
\pause \item XNAT upload applet (2008)
\pause \item XNAT archiving (2011)
\end{itemize}
\end{frame}

\section{The language}
\subsection{Operations}
\begin{frame}[fragile]{Operations}

\begin{verbatim}
// Assign a uniform Institution Name
(0008,0080) := "WUSM"
\end{verbatim}

\pause
\begin{verbatim}
// Delete the Institution Address
// and Frame of Reference UID
- (0008,0081)
- (0020,0052)
\end{verbatim}

\end{frame}

\begin{frame}[fragile]{Conditional operations}

\begin{verbatim}
// Use Referring Physician's Name
// to identify research group;
// put research group identifier
// into Clinical Trial Protocol ID
(0008,0090) = "Lee^Ann"   : (0012,0020) := "P-035"
(0008,0090) = "Hill^John" : (0012,0020) := "P-028"
\end{verbatim}

\pause
\begin{verbatim}
// Series with 3-digit numbers are derived data
(0020,0011) ~ "\\d\\d\\d" : \
  (0008,103e) := format["SECONDARY {0}", (0008,103e)]
\end{verbatim}

\end{frame}

\subsection{User-defined variables}
\begin{frame}[fragile]{User-defined variables}

\begin{verbatim}
// Allow the user to edit Patient ID
// Initial value is the pre-modification value
patientID := (0010,0020)
describe patientID "Subject ID"
(0010,0020) := patientID
\end{verbatim}

\end{frame}

\subsection{Built-in functions}
\begin{frame}[fragile]{Built-in functions}
\scriptsize
\begin{verbatim}
// Get visit ID from web service
// using Patient ID and Study Date
describe visurl hidden
visurl := format["http://localhost:3000/services/visitID?id={0}&date={1}", \
       urlEncode[(0010,0020)], urlEncode[(0008,0020)]]
describe visit hidden
visit := getURL[visurl]


// Generate new Study Description based on Patient ID, Visit ID, modality
describe studyDesc "Study Description"
studyDesc := format["{0}_{1}_{2}", (0010,0020), visit, (0008,0060)]
(0008,1030) := studyDesc
\end{verbatim}
\end{frame}

\begin{frame}[fragile]{Built-in functions: text formatting}
\begin{center}
\small
\begin{tabular}{ l p{1in} p{2in} }
Name & Arguments & Description \\ \hline

format & {\em format-string}, $v_1, v_2, \dots$ &
Formats the given values using the same syntax as java.text.MessageFormat \\

\pause
lowercase & {\em string} &
Converts all uppercase characters to lowercase \\

\pause
uppercase & {\em string} &
Converts all lowercase characters to uppercase \\

\pause
substring & {\em string, start\nobreakdash-index,
end\nobreakdash-index} &
Returns a substring, similar to \verb|s.substring(start,end)| \\

\end{tabular}
\end{center}

\pause
\hrule
\small
\begin{verbatim}
(0008,103e) := "SECONDARY {0} - {1}" (0008,103e), (0020,0011)
\end{verbatim}
is (deprecated) shorthand for
\small
\begin{verbatim}
(0008,103e) := format["SECONDARY {0} - {1}", \
    (0008,103e), (0020,0011)]
\end{verbatim}

\end{frame}

\begin{frame}[fragile]{Built-in functions: pattern matching}
\begin{center}
\small
\begin{tabular}{ l p{1in} p{2in} }
Name & Arguments & Description \\ \hline

replace & {\em string, target, \mbox{replacement}} &
Replaces all occurrences of \mbox{\em target} in the
given string with \mbox{\em replacement} \\

\pause
match & {\em value, pattern, i} &
Matches {\em value} against {\em pattern} and returns
the content of the $i$th matching group \\

\end{tabular}
\end{center}

\pause
\hrule
\vspace{2mm}
\footnotesize
\verb|match(value,pattern,i)| is comparable to:
\verb|java.util.regex.Pattern.compile(pattern).matcher(value).group(i)|
\end{frame}

\begin{frame}[fragile]{Built-in functions: web services}
\begin{center}
\small
\begin{tabular}{ l p{1in} p{2in} }
Name & Arguments & Description \\ \hline

getURL & {\em url} &
Retrieves the content of the resource at the given URL \\

\pause
urlEncode & {\em string} &
Encodes the given value so that it can be embedded in a URL \\

\end{tabular}
\end{center}

\pause
\hrule
\scriptsize
\begin{verbatim}
describe visurl hidden
visurl := format["http://localhost:3000/services/visitID?id={0}&date={1}", \
       urlEncode[(0010,0020)], urlEncode[(0008,0020)]]
visit := getURL[visurl]
\end{verbatim}

\end{frame}

\subsection{Generators}
\begin{frame}[fragile]{UID Generator}
Reassigning UIDs is easy:
\begin{verbatim}
// Reassign UIDs
(0020,000D) := new UID	// Study Instance UID
(0020,000E) := new UID	// Series Instance UID
(0008,0018) := new UID	// SOP Instance UID
(0008,1155) := new UID	// Referenced SOP Instance UID
\end{verbatim}
(Actually, reassigning UIDs is complicated. But the UID generator does
the right thing for you.)
\end{frame}

\subsection{Phases of execution}
\begin{frame}[fragile]{Phases of script execution}
\begin{enumerate}
\item Parsing (global)
  \begin{enumerate}[i]
    \pause
    \item Abstract Syntax Tree (AST) construction
    \pause
    \item Variable initialization
    \pause
    \item Variable value assignment
  \end{enumerate}
\pause
\item Application (object-by-object)
  \begin{enumerate}[i]
    \pause
    \item Value evaluation
    \pause
    \item Tag assignment
    \pause
    \item Object export
  \end{enumerate}
\end{enumerate}

\pause
Variable initialization from DICOM attributes is an ill-posed
problem. DicomBrowser joins values into a comma-separated list, while
the upload applet uses a single value from an arbitrary DICOM object.

\end{frame}

\subsection{Example: DIAN script for XNAT}
\begin{frame}[fragile]
\tiny
\begin{verbatim}
// Common DIAN anonymization script v. 001, 2009-08-21
// by Kevin Archie, karchie@wustl.edu
(0008,0080) = "Columbia University" : (0008,0080) := "DIAN_010" // Institution Name
(0008,0080) = "Hatch Ctr 1.5T @ Columbia" : (0008,0080) := "DIAN_010" 
(0008,0080) = "CCIR 3T" : (0008,0080) := "DIAN_011" 
(0008,0080) = "Center for Clinical Imaging Research" : (0008,0080) := "DIAN_011" 
(0008,0080) = "Washington University" : (0008,0080) := "DIAN_011" 
(0008,0080) = "UCLA BRAIN MAPPING CENTER" : (0008,0080) := "DIAN_035" 
(0008,0080) = "Brain Mapping" : (0008,0080) := "DIAN_035" 
(0008,0080) = "Ser.Nr. 58013" : (0008,0080) := "DIAN_035"
(0008,0080) = "Indiana University Radiology Associates" : (0008,0080) := "DIAN_037" 
(0008,0080) = "INDIANA UNIVERSITY RESEARCH" : (0008,0080) := "DIAN_037" 
(0008,0080) = "Martinos Center Bay 4" : (0008,0080) := "DIAN_094" 
(0008,0080) = "BROWN UNIVERSITY MRF" : (0008,0080) := "DIAN_941" 
(0008,0080) = "POW Medical Research Institute" : (0008,0080) := "DIAN_950"
(0008,0080) = "Diagnostic MRI Services" : (0008,0080) := "DIAN_950"
(0008,0080) = "Austin Health, Centre for PET" : (0008,0080) := project 
(0008,0080) = "<Full Institution Name>" : (0008,0080) := "DIAN_951"
(0008,0080) = "Brain Research Institute" : (0008,0080) := "DIAN_951"
(0008,0080) = "SKG Radiology Subiaco" : (0008,0080) := "DIAN_952"
(0008,0080) = "0ECU 0952" : (0008,0080) := "DIAN_952"
(0008,0080) = "0ECU0952" : (0008,0080) := "DIAN_952"
(0008,0080) = "The WA PET/Cyclotron Service" : (0008,0080) := "DIAN_952"
(0008,0080) = "National Hospital for Neurology & Neurosurgery" : (0008,0080) := "DIAN_953"
(0008,0080) = "nhnn" : (0008,0080) := "DIAN_953"
(0008,0080) = "NHNN" : (0008,0080) := "DIAN_953"
(0008,0080) = "Addenbrookes Cambridge" : (0008,0080) := "DIAN_953"
(0008,0080) = "Hospital" : (0008,0080) := "DIAN_Test"
\end{verbatim}
\end{frame}

\begin{frame}[fragile]
\tiny
\begin{verbatim}
- (0008,0050)     // Accession Number
- (0010,0021)     // Issuer of Patient ID
- (0010,0030)     // Patient's Birth Date
- (0010,0032)     // Patient's Birth Time
- (0010,0050)     // Patient's Insurance Plan Code SQ
- (0010,0101)     // Patient's Primary Language Code SQ
- (0010,1000)     // Other Patient IDs
- (0010,1001)     // Other Patient Names
- (0010,1002)     // Other Patient IDs SQ
- (0010,1005)     // Patient's Birth Name
- (0010,1010)     // Patient's Age
- (0010,1040)     // Patient's Address
- (0010,1060)     // Patient's Mother's Birth Name
- (0010,1080)     // Military Rank
- (0010,1081)     // Branch of Service
- (0010,1090)     // Medical Record Locator
- (0010,2000)     // Medical Alerts
- (0010,2110)     // Allergies
- (0010,2150)     // Country of Residence
- (0010,2152)     // Region of Residence
- (0010,2154)     // Patient's Telephone Numbers
- (0010,2160)     // Ethnic Group
- (0010,2180)     // Occupation
- (0010,21A0)     // Smoking Status
- (0010,21B0)     // Additional Patient History
- (0010,21C0)     // Pregnancy Status
- (0010,21D0)     // Last Menstrual Date
- (0010,21F0)     // Patient's Religious Preference
- (0010,2203)     // Patient's Sex Neutered
- (0010,2297)     // Responsible Person
- (0010,2298)     // Responsible Person Role
- (0010,2299)     // Responsible Organization
- (0038,0500)     // Patient State
- (0038,0100)     // Pertinent Documents SQ
\end{verbatim}
\end{frame}

\begin{frame}[fragile]
\tiny
\begin{Verbatim}[commandchars=\\\{\}]
(0012,0063) := "DIAN common deidentification v001"

session := \textcolor{red}{makeSessionLabel}[format["\{0\}_v##_\{1\}", subject, lowercase[modalityLabel]]]

// Use a regular expression to pull the visit ID out of the session label
visit := match[session,".*_(v[0-9]+)_.*",1]

// Don't show the visit field in the assign session variables page
describe visit hidden                           

// Use visit to set the named XNAT field
\textcolor{red}{export visit "xnat:experimentData/visit_id"}

// Set XNAT identifying fields
(0008,1030) := project
(0010,0010) := subject
(0010,0020) := session
\end{Verbatim}
\end{frame}

\section{In DicomBrowser}
\begin{frame}{DicomEdit in DicomBrowser}
DicomBrowser v1.5 (coming soon) includes new DicomEdit features
\begin{itemize}
  \item User-defined variables
  \item Built-in functions
\end{itemize}
\vspace{5mm}
\hrule
\vspace{5mm}
DicomBrowser v1.5 beta available via anonymous FTP at
\url{ftp://ftp.nrg.wustl.edu/pub/DicomBrowser}
\end{frame}

\section{In XNAT}
\subsection{Upload applet}
\begin{frame}{DicomEdit in XNAT upload applet}
The XNAT upload applet reads a project-specific DicomEdit script,
if available, from

\vspace{5mm}
\scriptsize
\nolinkurl{http://my.xnat.org/data/projects/Proj/resources/UPLOAD_CONFIG/dicom.das}
\vspace{5mm}

\normalsize
(As of XNAT 1.5, there is no web UI for setting that script.)

DicomEdit for XNAT includes some special features (makeSessionLabel and export).
\end{frame}

\begin{frame}{DicomEdit in XNAT archiving}

Starting in XNAT 1.6, XNAT will apply the project DicomEdit script to
all sessions as they are archived or moved from one project to
another.

\pause
There are some subtleties, so how this is configured might or might
not change from how the upload applet is configured.
\end{frame}

\section{The future}
\begin{frame}{Future features}
\begin{itemize}
\pause \item Sequence support (nested tags) --- especially important
for multiframe
\pause \item Tag patterns (or special support for vendor private tags)
\pause \item Better error messages and debugging support
\pause \item Value validity checking
\pause \item Integration/consolidation with MIRC Anonymizer language
\end{itemize}
\end{frame}

\end{document}
