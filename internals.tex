\documentclass[handout]{beamer}
% \documentclass{beamer}
\usetheme{Singapore}
\usepackage{enumerate,picture}
\title{DicomEdit internals}
\author{Kevin A. Archie}
\institute{Washington University Neuroinformatics Research Group}
\date{12 September 2011}

\begin{document}

\begin{frame}
\titlepage
\end{frame}

\section{Introduction}

\begin{frame}{What is DicomEdit?}
\begin{itemize}
\pause \item Small language for specifying changes to DICOM metadata
\pause \item a.k.a. ``anon script'' language
\pause \item Interpreter is a modular component:
\url{http://hg.xnat.org/dicomedit}
\pause \item Similar in spirit (but not syntax) to MIRC Anonymizer language
\end{itemize}
\end{frame}


\subsection{History}
\begin{frame}{History}
\begin{itemize}
\pause \item DicomBrowser (late 2006; v1.0 released 1 March 2007)
\pause \item Scriptable modifications (in DicomBrowser v1.0)
\pause \item DicomSummarize/DicomRemap (first command-line DicomEdit, early 2008)
\pause \item XNAT upload applet (2008)
\pause \item XNAT archiving (2011)
\end{itemize}
\end{frame}

\section{The implementation}
\subsection{Parsing:Overview}
\begin{frame}[fragile]
\centering
\begin{tabular}{cl}
script (byte stream) \\
$\downarrow$ & {\em lexical analysis} \\
token stream \\
$\downarrow$ & {\em parsing} \\
abstract syntax tree (AST) \\
$\downarrow$ & {\em AST parsing} \\
internal representation
\end{tabular}
\end{frame}

\subsection{Parsing: Implementation}
\begin{frame}
\begin{itemize}
\item Three parsing components (lexer, parser, AST parser)
 are written in EBNF for ANTLR
\pause \item What for what now?
\pause \item Extended Backus-Naur Form (EBNF) is a notation for representing (most) languages
\pause \item ANTLR is a set of tools for building interpreters, compilers, etc.
\pause \item ANTLR compiles its source files (EBNF, mostly) into Java (or C, C#, JavaScript, Python, Ruby, ...)
\pause \item {\em but if I had to start over...}
\end{frame}

% go look at the ANTLR source files
% points for extension:
%  * changes to elementary types happen at lexer level
%    (and will probably need Java type changes)
%    e.g.: comparison operators, tag expressions
%  * changes to syntax mostly happen at parser level
%    e.g.: variable value enumeration
% go look (briefly) at the ANTLR target files

% next: ScriptApplicator
% sets up the three parsers
% adds functions and generators to the AST parser
% see in particular: getTags(), getVariable(), getSortedVariables(),
%   apply()

\end{document}
